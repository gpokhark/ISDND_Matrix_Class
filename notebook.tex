
% Default to the notebook output style

    


% Inherit from the specified cell style.




    
\documentclass[11pt]{article}

    
    
    \usepackage[T1]{fontenc}
    % Nicer default font (+ math font) than Computer Modern for most use cases
    \usepackage{mathpazo}

    % Basic figure setup, for now with no caption control since it's done
    % automatically by Pandoc (which extracts ![](path) syntax from Markdown).
    \usepackage{graphicx}
    % We will generate all images so they have a width \maxwidth. This means
    % that they will get their normal width if they fit onto the page, but
    % are scaled down if they would overflow the margins.
    \makeatletter
    \def\maxwidth{\ifdim\Gin@nat@width>\linewidth\linewidth
    \else\Gin@nat@width\fi}
    \makeatother
    \let\Oldincludegraphics\includegraphics
    % Set max figure width to be 80% of text width, for now hardcoded.
    \renewcommand{\includegraphics}[1]{\Oldincludegraphics[width=.8\maxwidth]{#1}}
    % Ensure that by default, figures have no caption (until we provide a
    % proper Figure object with a Caption API and a way to capture that
    % in the conversion process - todo).
    \usepackage{caption}
    \DeclareCaptionLabelFormat{nolabel}{}
    \captionsetup{labelformat=nolabel}

    \usepackage{adjustbox} % Used to constrain images to a maximum size 
    \usepackage{xcolor} % Allow colors to be defined
    \usepackage{enumerate} % Needed for markdown enumerations to work
    \usepackage{geometry} % Used to adjust the document margins
    \usepackage{amsmath} % Equations
    \usepackage{amssymb} % Equations
    \usepackage{textcomp} % defines textquotesingle
    % Hack from http://tex.stackexchange.com/a/47451/13684:
    \AtBeginDocument{%
        \def\PYZsq{\textquotesingle}% Upright quotes in Pygmentized code
    }
    \usepackage{upquote} % Upright quotes for verbatim code
    \usepackage{eurosym} % defines \euro
    \usepackage[mathletters]{ucs} % Extended unicode (utf-8) support
    \usepackage[utf8x]{inputenc} % Allow utf-8 characters in the tex document
    \usepackage{fancyvrb} % verbatim replacement that allows latex
    \usepackage{grffile} % extends the file name processing of package graphics 
                         % to support a larger range 
    % The hyperref package gives us a pdf with properly built
    % internal navigation ('pdf bookmarks' for the table of contents,
    % internal cross-reference links, web links for URLs, etc.)
    \usepackage{hyperref}
    \usepackage{longtable} % longtable support required by pandoc >1.10
    \usepackage{booktabs}  % table support for pandoc > 1.12.2
    \usepackage[inline]{enumitem} % IRkernel/repr support (it uses the enumerate* environment)
    \usepackage[normalem]{ulem} % ulem is needed to support strikethroughs (\sout)
                                % normalem makes italics be italics, not underlines
    

    
    
    % Colors for the hyperref package
    \definecolor{urlcolor}{rgb}{0,.145,.698}
    \definecolor{linkcolor}{rgb}{.71,0.21,0.01}
    \definecolor{citecolor}{rgb}{.12,.54,.11}

    % ANSI colors
    \definecolor{ansi-black}{HTML}{3E424D}
    \definecolor{ansi-black-intense}{HTML}{282C36}
    \definecolor{ansi-red}{HTML}{E75C58}
    \definecolor{ansi-red-intense}{HTML}{B22B31}
    \definecolor{ansi-green}{HTML}{00A250}
    \definecolor{ansi-green-intense}{HTML}{007427}
    \definecolor{ansi-yellow}{HTML}{DDB62B}
    \definecolor{ansi-yellow-intense}{HTML}{B27D12}
    \definecolor{ansi-blue}{HTML}{208FFB}
    \definecolor{ansi-blue-intense}{HTML}{0065CA}
    \definecolor{ansi-magenta}{HTML}{D160C4}
    \definecolor{ansi-magenta-intense}{HTML}{A03196}
    \definecolor{ansi-cyan}{HTML}{60C6C8}
    \definecolor{ansi-cyan-intense}{HTML}{258F8F}
    \definecolor{ansi-white}{HTML}{C5C1B4}
    \definecolor{ansi-white-intense}{HTML}{A1A6B2}

    % commands and environments needed by pandoc snippets
    % extracted from the output of `pandoc -s`
    \providecommand{\tightlist}{%
      \setlength{\itemsep}{0pt}\setlength{\parskip}{0pt}}
    \DefineVerbatimEnvironment{Highlighting}{Verbatim}{commandchars=\\\{\}}
    % Add ',fontsize=\small' for more characters per line
    \newenvironment{Shaded}{}{}
    \newcommand{\KeywordTok}[1]{\textcolor[rgb]{0.00,0.44,0.13}{\textbf{{#1}}}}
    \newcommand{\DataTypeTok}[1]{\textcolor[rgb]{0.56,0.13,0.00}{{#1}}}
    \newcommand{\DecValTok}[1]{\textcolor[rgb]{0.25,0.63,0.44}{{#1}}}
    \newcommand{\BaseNTok}[1]{\textcolor[rgb]{0.25,0.63,0.44}{{#1}}}
    \newcommand{\FloatTok}[1]{\textcolor[rgb]{0.25,0.63,0.44}{{#1}}}
    \newcommand{\CharTok}[1]{\textcolor[rgb]{0.25,0.44,0.63}{{#1}}}
    \newcommand{\StringTok}[1]{\textcolor[rgb]{0.25,0.44,0.63}{{#1}}}
    \newcommand{\CommentTok}[1]{\textcolor[rgb]{0.38,0.63,0.69}{\textit{{#1}}}}
    \newcommand{\OtherTok}[1]{\textcolor[rgb]{0.00,0.44,0.13}{{#1}}}
    \newcommand{\AlertTok}[1]{\textcolor[rgb]{1.00,0.00,0.00}{\textbf{{#1}}}}
    \newcommand{\FunctionTok}[1]{\textcolor[rgb]{0.02,0.16,0.49}{{#1}}}
    \newcommand{\RegionMarkerTok}[1]{{#1}}
    \newcommand{\ErrorTok}[1]{\textcolor[rgb]{1.00,0.00,0.00}{\textbf{{#1}}}}
    \newcommand{\NormalTok}[1]{{#1}}
    
    % Additional commands for more recent versions of Pandoc
    \newcommand{\ConstantTok}[1]{\textcolor[rgb]{0.53,0.00,0.00}{{#1}}}
    \newcommand{\SpecialCharTok}[1]{\textcolor[rgb]{0.25,0.44,0.63}{{#1}}}
    \newcommand{\VerbatimStringTok}[1]{\textcolor[rgb]{0.25,0.44,0.63}{{#1}}}
    \newcommand{\SpecialStringTok}[1]{\textcolor[rgb]{0.73,0.40,0.53}{{#1}}}
    \newcommand{\ImportTok}[1]{{#1}}
    \newcommand{\DocumentationTok}[1]{\textcolor[rgb]{0.73,0.13,0.13}{\textit{{#1}}}}
    \newcommand{\AnnotationTok}[1]{\textcolor[rgb]{0.38,0.63,0.69}{\textbf{\textit{{#1}}}}}
    \newcommand{\CommentVarTok}[1]{\textcolor[rgb]{0.38,0.63,0.69}{\textbf{\textit{{#1}}}}}
    \newcommand{\VariableTok}[1]{\textcolor[rgb]{0.10,0.09,0.49}{{#1}}}
    \newcommand{\ControlFlowTok}[1]{\textcolor[rgb]{0.00,0.44,0.13}{\textbf{{#1}}}}
    \newcommand{\OperatorTok}[1]{\textcolor[rgb]{0.40,0.40,0.40}{{#1}}}
    \newcommand{\BuiltInTok}[1]{{#1}}
    \newcommand{\ExtensionTok}[1]{{#1}}
    \newcommand{\PreprocessorTok}[1]{\textcolor[rgb]{0.74,0.48,0.00}{{#1}}}
    \newcommand{\AttributeTok}[1]{\textcolor[rgb]{0.49,0.56,0.16}{{#1}}}
    \newcommand{\InformationTok}[1]{\textcolor[rgb]{0.38,0.63,0.69}{\textbf{\textit{{#1}}}}}
    \newcommand{\WarningTok}[1]{\textcolor[rgb]{0.38,0.63,0.69}{\textbf{\textit{{#1}}}}}
    
    
    % Define a nice break command that doesn't care if a line doesn't already
    % exist.
    \def\br{\hspace*{\fill} \\* }
    % Math Jax compatability definitions
    \def\gt{>}
    \def\lt{<}
    % Document parameters
    \title{matrix\_cheat\_sheet}
    
    
    

    % Pygments definitions
    
\makeatletter
\def\PY@reset{\let\PY@it=\relax \let\PY@bf=\relax%
    \let\PY@ul=\relax \let\PY@tc=\relax%
    \let\PY@bc=\relax \let\PY@ff=\relax}
\def\PY@tok#1{\csname PY@tok@#1\endcsname}
\def\PY@toks#1+{\ifx\relax#1\empty\else%
    \PY@tok{#1}\expandafter\PY@toks\fi}
\def\PY@do#1{\PY@bc{\PY@tc{\PY@ul{%
    \PY@it{\PY@bf{\PY@ff{#1}}}}}}}
\def\PY#1#2{\PY@reset\PY@toks#1+\relax+\PY@do{#2}}

\expandafter\def\csname PY@tok@w\endcsname{\def\PY@tc##1{\textcolor[rgb]{0.73,0.73,0.73}{##1}}}
\expandafter\def\csname PY@tok@c\endcsname{\let\PY@it=\textit\def\PY@tc##1{\textcolor[rgb]{0.25,0.50,0.50}{##1}}}
\expandafter\def\csname PY@tok@cp\endcsname{\def\PY@tc##1{\textcolor[rgb]{0.74,0.48,0.00}{##1}}}
\expandafter\def\csname PY@tok@k\endcsname{\let\PY@bf=\textbf\def\PY@tc##1{\textcolor[rgb]{0.00,0.50,0.00}{##1}}}
\expandafter\def\csname PY@tok@kp\endcsname{\def\PY@tc##1{\textcolor[rgb]{0.00,0.50,0.00}{##1}}}
\expandafter\def\csname PY@tok@kt\endcsname{\def\PY@tc##1{\textcolor[rgb]{0.69,0.00,0.25}{##1}}}
\expandafter\def\csname PY@tok@o\endcsname{\def\PY@tc##1{\textcolor[rgb]{0.40,0.40,0.40}{##1}}}
\expandafter\def\csname PY@tok@ow\endcsname{\let\PY@bf=\textbf\def\PY@tc##1{\textcolor[rgb]{0.67,0.13,1.00}{##1}}}
\expandafter\def\csname PY@tok@nb\endcsname{\def\PY@tc##1{\textcolor[rgb]{0.00,0.50,0.00}{##1}}}
\expandafter\def\csname PY@tok@nf\endcsname{\def\PY@tc##1{\textcolor[rgb]{0.00,0.00,1.00}{##1}}}
\expandafter\def\csname PY@tok@nc\endcsname{\let\PY@bf=\textbf\def\PY@tc##1{\textcolor[rgb]{0.00,0.00,1.00}{##1}}}
\expandafter\def\csname PY@tok@nn\endcsname{\let\PY@bf=\textbf\def\PY@tc##1{\textcolor[rgb]{0.00,0.00,1.00}{##1}}}
\expandafter\def\csname PY@tok@ne\endcsname{\let\PY@bf=\textbf\def\PY@tc##1{\textcolor[rgb]{0.82,0.25,0.23}{##1}}}
\expandafter\def\csname PY@tok@nv\endcsname{\def\PY@tc##1{\textcolor[rgb]{0.10,0.09,0.49}{##1}}}
\expandafter\def\csname PY@tok@no\endcsname{\def\PY@tc##1{\textcolor[rgb]{0.53,0.00,0.00}{##1}}}
\expandafter\def\csname PY@tok@nl\endcsname{\def\PY@tc##1{\textcolor[rgb]{0.63,0.63,0.00}{##1}}}
\expandafter\def\csname PY@tok@ni\endcsname{\let\PY@bf=\textbf\def\PY@tc##1{\textcolor[rgb]{0.60,0.60,0.60}{##1}}}
\expandafter\def\csname PY@tok@na\endcsname{\def\PY@tc##1{\textcolor[rgb]{0.49,0.56,0.16}{##1}}}
\expandafter\def\csname PY@tok@nt\endcsname{\let\PY@bf=\textbf\def\PY@tc##1{\textcolor[rgb]{0.00,0.50,0.00}{##1}}}
\expandafter\def\csname PY@tok@nd\endcsname{\def\PY@tc##1{\textcolor[rgb]{0.67,0.13,1.00}{##1}}}
\expandafter\def\csname PY@tok@s\endcsname{\def\PY@tc##1{\textcolor[rgb]{0.73,0.13,0.13}{##1}}}
\expandafter\def\csname PY@tok@sd\endcsname{\let\PY@it=\textit\def\PY@tc##1{\textcolor[rgb]{0.73,0.13,0.13}{##1}}}
\expandafter\def\csname PY@tok@si\endcsname{\let\PY@bf=\textbf\def\PY@tc##1{\textcolor[rgb]{0.73,0.40,0.53}{##1}}}
\expandafter\def\csname PY@tok@se\endcsname{\let\PY@bf=\textbf\def\PY@tc##1{\textcolor[rgb]{0.73,0.40,0.13}{##1}}}
\expandafter\def\csname PY@tok@sr\endcsname{\def\PY@tc##1{\textcolor[rgb]{0.73,0.40,0.53}{##1}}}
\expandafter\def\csname PY@tok@ss\endcsname{\def\PY@tc##1{\textcolor[rgb]{0.10,0.09,0.49}{##1}}}
\expandafter\def\csname PY@tok@sx\endcsname{\def\PY@tc##1{\textcolor[rgb]{0.00,0.50,0.00}{##1}}}
\expandafter\def\csname PY@tok@m\endcsname{\def\PY@tc##1{\textcolor[rgb]{0.40,0.40,0.40}{##1}}}
\expandafter\def\csname PY@tok@gh\endcsname{\let\PY@bf=\textbf\def\PY@tc##1{\textcolor[rgb]{0.00,0.00,0.50}{##1}}}
\expandafter\def\csname PY@tok@gu\endcsname{\let\PY@bf=\textbf\def\PY@tc##1{\textcolor[rgb]{0.50,0.00,0.50}{##1}}}
\expandafter\def\csname PY@tok@gd\endcsname{\def\PY@tc##1{\textcolor[rgb]{0.63,0.00,0.00}{##1}}}
\expandafter\def\csname PY@tok@gi\endcsname{\def\PY@tc##1{\textcolor[rgb]{0.00,0.63,0.00}{##1}}}
\expandafter\def\csname PY@tok@gr\endcsname{\def\PY@tc##1{\textcolor[rgb]{1.00,0.00,0.00}{##1}}}
\expandafter\def\csname PY@tok@ge\endcsname{\let\PY@it=\textit}
\expandafter\def\csname PY@tok@gs\endcsname{\let\PY@bf=\textbf}
\expandafter\def\csname PY@tok@gp\endcsname{\let\PY@bf=\textbf\def\PY@tc##1{\textcolor[rgb]{0.00,0.00,0.50}{##1}}}
\expandafter\def\csname PY@tok@go\endcsname{\def\PY@tc##1{\textcolor[rgb]{0.53,0.53,0.53}{##1}}}
\expandafter\def\csname PY@tok@gt\endcsname{\def\PY@tc##1{\textcolor[rgb]{0.00,0.27,0.87}{##1}}}
\expandafter\def\csname PY@tok@err\endcsname{\def\PY@bc##1{\setlength{\fboxsep}{0pt}\fcolorbox[rgb]{1.00,0.00,0.00}{1,1,1}{\strut ##1}}}
\expandafter\def\csname PY@tok@kc\endcsname{\let\PY@bf=\textbf\def\PY@tc##1{\textcolor[rgb]{0.00,0.50,0.00}{##1}}}
\expandafter\def\csname PY@tok@kd\endcsname{\let\PY@bf=\textbf\def\PY@tc##1{\textcolor[rgb]{0.00,0.50,0.00}{##1}}}
\expandafter\def\csname PY@tok@kn\endcsname{\let\PY@bf=\textbf\def\PY@tc##1{\textcolor[rgb]{0.00,0.50,0.00}{##1}}}
\expandafter\def\csname PY@tok@kr\endcsname{\let\PY@bf=\textbf\def\PY@tc##1{\textcolor[rgb]{0.00,0.50,0.00}{##1}}}
\expandafter\def\csname PY@tok@bp\endcsname{\def\PY@tc##1{\textcolor[rgb]{0.00,0.50,0.00}{##1}}}
\expandafter\def\csname PY@tok@fm\endcsname{\def\PY@tc##1{\textcolor[rgb]{0.00,0.00,1.00}{##1}}}
\expandafter\def\csname PY@tok@vc\endcsname{\def\PY@tc##1{\textcolor[rgb]{0.10,0.09,0.49}{##1}}}
\expandafter\def\csname PY@tok@vg\endcsname{\def\PY@tc##1{\textcolor[rgb]{0.10,0.09,0.49}{##1}}}
\expandafter\def\csname PY@tok@vi\endcsname{\def\PY@tc##1{\textcolor[rgb]{0.10,0.09,0.49}{##1}}}
\expandafter\def\csname PY@tok@vm\endcsname{\def\PY@tc##1{\textcolor[rgb]{0.10,0.09,0.49}{##1}}}
\expandafter\def\csname PY@tok@sa\endcsname{\def\PY@tc##1{\textcolor[rgb]{0.73,0.13,0.13}{##1}}}
\expandafter\def\csname PY@tok@sb\endcsname{\def\PY@tc##1{\textcolor[rgb]{0.73,0.13,0.13}{##1}}}
\expandafter\def\csname PY@tok@sc\endcsname{\def\PY@tc##1{\textcolor[rgb]{0.73,0.13,0.13}{##1}}}
\expandafter\def\csname PY@tok@dl\endcsname{\def\PY@tc##1{\textcolor[rgb]{0.73,0.13,0.13}{##1}}}
\expandafter\def\csname PY@tok@s2\endcsname{\def\PY@tc##1{\textcolor[rgb]{0.73,0.13,0.13}{##1}}}
\expandafter\def\csname PY@tok@sh\endcsname{\def\PY@tc##1{\textcolor[rgb]{0.73,0.13,0.13}{##1}}}
\expandafter\def\csname PY@tok@s1\endcsname{\def\PY@tc##1{\textcolor[rgb]{0.73,0.13,0.13}{##1}}}
\expandafter\def\csname PY@tok@mb\endcsname{\def\PY@tc##1{\textcolor[rgb]{0.40,0.40,0.40}{##1}}}
\expandafter\def\csname PY@tok@mf\endcsname{\def\PY@tc##1{\textcolor[rgb]{0.40,0.40,0.40}{##1}}}
\expandafter\def\csname PY@tok@mh\endcsname{\def\PY@tc##1{\textcolor[rgb]{0.40,0.40,0.40}{##1}}}
\expandafter\def\csname PY@tok@mi\endcsname{\def\PY@tc##1{\textcolor[rgb]{0.40,0.40,0.40}{##1}}}
\expandafter\def\csname PY@tok@il\endcsname{\def\PY@tc##1{\textcolor[rgb]{0.40,0.40,0.40}{##1}}}
\expandafter\def\csname PY@tok@mo\endcsname{\def\PY@tc##1{\textcolor[rgb]{0.40,0.40,0.40}{##1}}}
\expandafter\def\csname PY@tok@ch\endcsname{\let\PY@it=\textit\def\PY@tc##1{\textcolor[rgb]{0.25,0.50,0.50}{##1}}}
\expandafter\def\csname PY@tok@cm\endcsname{\let\PY@it=\textit\def\PY@tc##1{\textcolor[rgb]{0.25,0.50,0.50}{##1}}}
\expandafter\def\csname PY@tok@cpf\endcsname{\let\PY@it=\textit\def\PY@tc##1{\textcolor[rgb]{0.25,0.50,0.50}{##1}}}
\expandafter\def\csname PY@tok@c1\endcsname{\let\PY@it=\textit\def\PY@tc##1{\textcolor[rgb]{0.25,0.50,0.50}{##1}}}
\expandafter\def\csname PY@tok@cs\endcsname{\let\PY@it=\textit\def\PY@tc##1{\textcolor[rgb]{0.25,0.50,0.50}{##1}}}

\def\PYZbs{\char`\\}
\def\PYZus{\char`\_}
\def\PYZob{\char`\{}
\def\PYZcb{\char`\}}
\def\PYZca{\char`\^}
\def\PYZam{\char`\&}
\def\PYZlt{\char`\<}
\def\PYZgt{\char`\>}
\def\PYZsh{\char`\#}
\def\PYZpc{\char`\%}
\def\PYZdl{\char`\$}
\def\PYZhy{\char`\-}
\def\PYZsq{\char`\'}
\def\PYZdq{\char`\"}
\def\PYZti{\char`\~}
% for compatibility with earlier versions
\def\PYZat{@}
\def\PYZlb{[}
\def\PYZrb{]}
\makeatother


    % Exact colors from NB
    \definecolor{incolor}{rgb}{0.0, 0.0, 0.5}
    \definecolor{outcolor}{rgb}{0.545, 0.0, 0.0}



    
    % Prevent overflowing lines due to hard-to-break entities
    \sloppy 
    % Setup hyperref package
    \hypersetup{
      breaklinks=true,  % so long urls are correctly broken across lines
      colorlinks=true,
      urlcolor=urlcolor,
      linkcolor=linkcolor,
      citecolor=citecolor,
      }
    % Slightly bigger margins than the latex defaults
    
    \geometry{verbose,tmargin=1in,bmargin=1in,lmargin=1in,rmargin=1in}
    
    

    \begin{document}
    
    
    \maketitle
    
    

    
    \hypertarget{matrix-cheat-sheet}{%
\section{Matrix Cheat Sheet}\label{matrix-cheat-sheet}}

\hypertarget{glossary}{%
\subsection{Glossary}\label{glossary}}

\begin{itemize}
\tightlist
\item
  \textbf{Scalar} - A scalar is just a number like 1.2, -5, 0, or 239.
  When we use the word \textbf{scalar} it's usually to highlight that we
  are \emph{not} talking about a \textbf{vector} or a \textbf{matrix}.
\item
  \textbf{Matrix} - A matrix is a rectangular grid of numbers. For
  example, this is a matrix with \textbf{2 rows} and \textbf{3 columns}
  so we would call it a \(2\times 3\) ``2 by 3'' matrix.:
\end{itemize}

\[\begin{bmatrix}1.5 & -9.2 & 0 \\
5.4 & 7 & 2.2\end{bmatrix}\]

\begin{itemize}
\item
  \textbf{Row / Column} - These terms describe the horizontal (row) and
  vertical (column) sequences of numbers in a matrix. For example, the
  first row in the matrix above is
  \(\begin{bmatrix}1.5 & -9.2 & 0\end{bmatrix}\).
\item
  \textbf{Vector} - A vector is a matrix where either the width or the
  height is 1. When the height is 1, it's called a \textbf{row vector}.
  When the width is 1 it's called a \textbf{column vector}.
\item
  \textbf{Matrix Element} - The \textbf{element} in the first row and
  first column of the matrix given above is the number \(1.5\)
\item
  \textbf{Square Matrix} - A matrix is square when its height is equal
  to its width.
\item
  \textbf{Main Diagonal} - The main diagonal of a \textbf{square matrix}
  is the sequence of \textbf{elements} from the top left to bottom
  right. For the matrix below the main diagonal refers to the numbers 2
  and 6.
\end{itemize}

\[\begin{bmatrix}2 & 9 \\ -4 & 6 \end{bmatrix} \] * \textbf{Identity
Matrix} - This is a special \textbf{square matrix} where all of the
\textbf{elements} are equal to zero except those on the \textbf{main
diagonal}, which are equal to 1. This is a \(3\times 3\) identity
matrix:

\[\begin{bmatrix}1 & 0 & 0 \\ 0 & 1 & 0 \\ 0 & 0 & 1\end{bmatrix}\]

\hypertarget{matrix-notation}{%
\subsection{Matrix Notation}\label{matrix-notation}}

\hypertarget{indexing-with-mathbfa_ij}{%
\subsubsection{\texorpdfstring{Indexing with
\(\mathbf{A}_{ij}\)}{Indexing with \textbackslash{}mathbf\{A\}\_\{ij\}}}\label{indexing-with-mathbfa_ij}}

The numbers \(i\) and \(j\) are used to refer to the row number and
column number of a matrix (respectively).

\textbf{NOTE}: The top left element of a matrix is typically given by
\(\mathbf{A}_{11}\) and \textbf{not} \(\mathbf{A}_{00}\).**

    \begin{Verbatim}[commandchars=\\\{\}]
{\color{incolor}In [{\color{incolor} }]:} \PY{c+c1}{\PYZsh{} python demo}
        
        \PY{n}{A} \PY{o}{=} \PY{p}{[}
            \PY{p}{[}\PY{l+s+s2}{\PYZdq{}}\PY{l+s+s2}{A\PYZus{}1,1}\PY{l+s+s2}{\PYZdq{}}\PY{p}{,} \PY{l+s+s2}{\PYZdq{}}\PY{l+s+s2}{A\PYZus{}1,2}\PY{l+s+s2}{\PYZdq{}}\PY{p}{,} \PY{l+s+s2}{\PYZdq{}}\PY{l+s+s2}{A\PYZus{}1,3}\PY{l+s+s2}{\PYZdq{}}\PY{p}{]}\PY{p}{,}
            \PY{p}{[}\PY{l+s+s2}{\PYZdq{}}\PY{l+s+s2}{A\PYZus{}2,1}\PY{l+s+s2}{\PYZdq{}}\PY{p}{,} \PY{l+s+s2}{\PYZdq{}}\PY{l+s+s2}{A\PYZus{}2,2}\PY{l+s+s2}{\PYZdq{}}\PY{p}{,} \PY{l+s+s2}{\PYZdq{}}\PY{l+s+s2}{A\PYZus{}2,3}\PY{l+s+s2}{\PYZdq{}}\PY{p}{]}\PY{p}{,}
            \PY{p}{[}\PY{l+s+s2}{\PYZdq{}}\PY{l+s+s2}{A\PYZus{}3,1}\PY{l+s+s2}{\PYZdq{}}\PY{p}{,} \PY{l+s+s2}{\PYZdq{}}\PY{l+s+s2}{A\PYZus{}3,2}\PY{l+s+s2}{\PYZdq{}}\PY{p}{,} \PY{l+s+s2}{\PYZdq{}}\PY{l+s+s2}{A\PYZus{}3,3}\PY{l+s+s2}{\PYZdq{}}\PY{p}{]}\PY{p}{,}
            \PY{p}{[}\PY{l+s+s2}{\PYZdq{}}\PY{l+s+s2}{A\PYZus{}4,1}\PY{l+s+s2}{\PYZdq{}}\PY{p}{,} \PY{l+s+s2}{\PYZdq{}}\PY{l+s+s2}{A\PYZus{}4,2}\PY{l+s+s2}{\PYZdq{}}\PY{p}{,} \PY{l+s+s2}{\PYZdq{}}\PY{l+s+s2}{A\PYZus{}4,3}\PY{l+s+s2}{\PYZdq{}}\PY{p}{]}\PY{p}{,}
        \PY{p}{]}
        
        \PY{k}{print}\PY{p}{(}\PY{l+s+s2}{\PYZdq{}}\PY{l+s+s2}{the bottom right entry in the A matrix is:}\PY{l+s+s2}{\PYZdq{}}\PY{p}{,} \PY{n}{A}\PY{p}{[}\PY{l+m+mi}{3}\PY{p}{]}\PY{p}{[}\PY{l+m+mi}{2}\PY{p}{]}\PY{p}{)}
\end{Verbatim}


    \hypertarget{summation-with-sum}{%
\subsubsection{\texorpdfstring{Summation with
\(\sum\)}{Summation with \textbackslash{}sum}}\label{summation-with-sum}}

Summation is best described by example.

\[\sum_{i=1}^n a_{nn}\]

This equation can be read as ``The sum from i equals one to n of the
matrix element at row \(n\), column \(n\).''

When you see \(\sum\) you should think ``for loop''. The code below
demonstrates the following mathematical equation:

\[\sum_{i=1}^n \sum_{j=1}^m a_{ij}\]

    \begin{Verbatim}[commandchars=\\\{\}]
{\color{incolor}In [{\color{incolor} }]:} \PY{k}{def} \PY{n+nf}{sum\PYZus{}all\PYZus{}matrix\PYZus{}elements}\PY{p}{(}\PY{n}{A}\PY{p}{)}\PY{p}{:}
            \PY{l+s+sd}{\PYZdq{}\PYZdq{}\PYZdq{}}
        \PY{l+s+sd}{    Computes the sum of ALL elements in some matrix A.}
        \PY{l+s+sd}{    \PYZdq{}\PYZdq{}\PYZdq{}}
            \PY{n}{n} \PY{o}{=} \PY{n+nb}{len}\PY{p}{(}\PY{n}{A}\PY{p}{)}
            \PY{n}{m} \PY{o}{=} \PY{n+nb}{len}\PY{p}{(}\PY{n}{A}\PY{p}{[}\PY{l+m+mi}{0}\PY{p}{]}\PY{p}{)}
            \PY{n}{total} \PY{o}{=} \PY{l+m+mf}{0.0}
            \PY{k}{for} \PY{n}{i} \PY{o+ow}{in} \PY{n+nb}{range}\PY{p}{(}\PY{n}{n}\PY{p}{)}\PY{p}{:}
                \PY{k}{for} \PY{n}{j} \PY{o+ow}{in} \PY{n+nb}{range}\PY{p}{(}\PY{n}{m}\PY{p}{)}\PY{p}{:}
                    \PY{n}{total} \PY{o}{=} \PY{n}{total} \PY{o}{+} \PY{n}{A}\PY{p}{[}\PY{n}{i}\PY{p}{]}\PY{p}{[}\PY{n}{j}\PY{p}{]}
            \PY{k}{return} \PY{n}{total}
        
        \PY{n}{example\PYZus{}matrix} \PY{o}{=} \PY{p}{[}
            \PY{p}{[}\PY{l+m+mi}{1}\PY{p}{,} \PY{l+m+mi}{3}\PY{p}{,} \PY{l+m+mi}{5}\PY{p}{]}\PY{p}{,}
            \PY{p}{[}\PY{l+m+mi}{4}\PY{p}{,} \PY{l+m+mi}{2}\PY{p}{,} \PY{l+m+mi}{2}\PY{p}{]}\PY{p}{,}
        \PY{p}{]}
        
        \PY{k}{print}\PY{p}{(}\PY{n}{sum\PYZus{}all\PYZus{}matrix\PYZus{}elements}\PY{p}{(}\PY{n}{example\PYZus{}matrix}\PY{p}{)}\PY{p}{)}
\end{Verbatim}


    \hypertarget{matrix-equations}{%
\subsection{Matrix Equations}\label{matrix-equations}}

\hypertarget{addition-subtraction}{%
\subsubsection{Addition / Subtraction}\label{addition-subtraction}}

Matrix addition and subtraction is an element by element operation. Two
matrices must have the same dimensions in order to be added or
subtracted.

\[\mathbf{A} + \mathbf{B} = \begin{bmatrix}
a_{11} & a_{12}  & \dots  & a_{1n}\\ 
a_{21} & a_{22}  & \dots &a_{2n} \\ 
\vdots & \vdots & \ddots  & \vdots \\ 
a_{m1} & a_{m2}  & \dots  & a_{mn}
\end{bmatrix} + \begin{bmatrix}
b_{11} & b_{12}  & \dots  & b_{1n}\\ 
b_{21} & b_{22}  & \dots &b_{2n} \\ 
\vdots & \vdots  & \ddots  & \vdots \\ 
b_{m1} & b_{m2}  &  \dots & b_{mn}
\end{bmatrix}\]

\[= \begin{bmatrix}
a_{11}+b_{11} & a_{12}+b_{12}  & \dots  & a_{1n}+b_{1n}\\ 
a_{21}+b_{21} & a_{22}+b_{22}  & \dots & a_{2n}+b_{2n} \\ 
\vdots & \vdots  & \ddots  & \vdots \\ 
a_{m1}+b_{m1} & a_{m2}+b_{m2}  &  \dots & a_{mn}+b_{mn}
\end{bmatrix}\]

\hypertarget{scalar-multiplication}{%
\subsubsection{Scalar Multiplication}\label{scalar-multiplication}}

When multiplying a matrix \(\mathbf{A}\) by a scalar \(c\), all of the
entries in \(\mathbf{A}\) are multiplied by \(c\):

\[c\mathbf{A} = c \begin{bmatrix}a_{11} & a_{12}  & \dots  & a_{1n}\\ 
a_{21} & a_{22}  & \dots &a_{2n} \\ 
\vdots & \vdots & \ddots  & \vdots \\ 
a_{m1} & a_{m2}  & \dots  & a_{mn}
\end{bmatrix} = \begin{bmatrix} ca_{11} & ca_{12}  & \dots  & ca_{1n}\\ 
ca_{21} & ca_{22}  & \dots &ca_{2n} \\ 
\vdots & \vdots & \ddots  & \vdots \\ 
ca_{m1} & ca_{m2}  & \dots  & ca_{mn}\end{bmatrix}\]

\hypertarget{matrix-multiplication}{%
\subsubsection{Matrix Multiplication}\label{matrix-multiplication}}

Multiplication of Matrix \(\mathbf{A}\) with matrix \(\mathbf{B}\) is
only possible if the width of \(\mathbf{A}\) is equal to the height of
\(\mathbf{B}\)

If \(\mathbf{A}\) is an \(m \times n\) matrix and \(\mathbf{B}\) is an
\(n \times p\) matrix, their product \(\mathbf{AB}\) is an
\(m \times p\) matrix.

When multiplying two matrices, we can calculate the value of the element
at row \(i\) and column \(j\) with the following equation:

\[(\mathbf{AB})_{ij} = \sum_{k=1}^n a_{ik}b_{kj}\]

\hypertarget{transpose}{%
\subsubsection{Transpose}\label{transpose}}

The transpose of a matrix \(\mathbf{A}\) is given by \(\mathbf{A^T}\)
and can be thought of in several ways:

\begin{itemize}
\tightlist
\item
  The \textbf{rows} of \(\mathbf{A^T}\) are the \textbf{columns} of
  \(\mathbf{A}\).
\item
  The \textbf{columns} of \(\mathbf{A^T}\) are the \textbf{rows} of
  \(\mathbf{A}\).
\end{itemize}

Mathematically, the element at row \(i\) and column \(j\) of the
transpose is given by:

\[[\mathbf{A^T}]_{ij} = [\mathbf{A}]_{ji}\]

\hypertarget{trace}{%
\subsubsection{Trace}\label{trace}}

The trace of an \(n\times n\) square matrix \(\mathbf{A}\) is the sum of
the elements on the \textbf{main diagonal} of the matrix.

\[\text{tr}\left(\mathbf{A}\right) = \sum_{i=1}^n a_{ii} = a_{11} + a_{22} + \dots + a_{nn}\]

\hypertarget{determinant}{%
\subsubsection{Determinant}\label{determinant}}

The determinant is a useful value when describing a matrix. It can be
denoted in one of three ways:

\begin{enumerate}
\def\labelenumi{\arabic{enumi}.}
\tightlist
\item
  \(\text{det } \left(\mathbf{A}\right)\)
\item
  \(\text{det } \mathbf{A}\)
\item
  \(|\mathbf{A}|\)
\end{enumerate}

\textbf{1x1 Matrices}

The determinant of a \(1\times1\) matrix is just the value of the
matrice's only element. For example if
\(\mathbf{A} = \begin{vmatrix}4\end{vmatrix}\), then the determinant of
\(\mathbf{A}\) is given by:

\[\begin{vmatrix}\mathbf{A}\end{vmatrix} = 4\]

\textbf{2x2 Matrices}

The determinant of a \(2\times2\) matrix is given by:

\[\begin{vmatrix}\mathbf{A}\end{vmatrix} = \begin{vmatrix}
a & b \\
c & d \end{vmatrix} = ad - bc
\]

\textbf{Larger Matrices}

You will not be required to calculate the determinant for larger
matrices. If you are interested in learning more you should look at the
Wikipedia article:
\href{https://en.wikipedia.org/wiki/Determinant}{Determinant}.

\hypertarget{inverse}{%
\subsubsection{Inverse}\label{inverse}}

The inverse of a matrix \(\mathbf{A}\) is given by \(\mathbf{A^{-1}}\)

A matrix \(\mathbf{A}\) is \textbf{invertible} if there exists a matrix
\(\mathbf{B}\) such that the product of \(\mathbf{A}\) and
\(\mathbf{B}\) is the \textbf{identity matrix} \(\mathbf{I}\):

\[\mathbf{AB} = \mathbf{BA} = \mathbf{I}\]

\textbf{1x1 Matrices}

For a \(1\times1\) matrix with a single element with value \(a\), the
inverse is simlpy \(\frac{1}{a}\)

\textbf{2x2 Matrices}

The inverse of a \(2\times 2\) matrix is given by the following
equation:

\[\mathbf{A}^{-1} = \frac{1}{\text{det }\mathbf{A}} \left[\left(\text{tr } \mathbf{A}\right) \mathbf{I} - \mathbf{A}\right]\]

\textbf{Larger Matrices}

You will not be required to invert larger matrices. If you are
interested in learning more you should look at the Wikipedia article
\href{https://en.wikipedia.org/wiki/Invertible_matrix}{Invertible
Matrix}.


    % Add a bibliography block to the postdoc
    
    
    
    \end{document}
